\chapter{Introduction Chapter}\label{ch:intro-chapter}
\section{Inversion of Control}\label{sec:inv-of-control}
The Inversion of Control pattern can be seen in many programming frameworks, often being a defining characteristic of them.
A programmer often controls when his code is called.
However, when using a framework, control is handed over to said framework.
This is Inversion of Control, also known as "The Hollywood Principle" (don't call us, we'll call you).
It is a key difference between a library and a framework, as in a library, you call the functions, while many frameworks invert the control.\parencite{fowler_bliki_2005}
\newline
\section{Dependency Injection}\label{sec:dep-injection}
A common issue in programming is how to wire together different elements.
If you have a team building a web controller and another team building a database backend, how do you get them to work together?
For Dependency Injection, the basic idea is that a separate object, an "assembler", populates a field in a class with the appropriate implementation of an interface.
There are three main ways of doing this: Constructor Injection, Setter Injection and Interface Injection.
\subsection{Constructor Injection}\label{subsec:cons-injection}
Like the name implies, the injection takes place in the constructor in such a case.
\begin{longlisting}
    \begin{minted}{java}
        public class MovieLister {
            private MovieFinder finder;
            public MovieLister(MovieFinder finder) {
                this.finder = finder;
            }
        }
    \end{minted}
    \caption[Dependency Injection Example]{A code example showing dependency injection via constructor.}
    \label{lst:ci-example}
\end{longlisting}
Note that MovieFinder is an interface in this example.
All classes that implement said interface can be passed into the constructor and are injected into the class that way.\parencite{fowler_inversion_2004}

